\stopcounter
\begin{frame}{Origem}
  O \textbf{Freeze-Tag Problem (FTP)} surge como um problema de robótica de enxame em 2002~\cite{Arkin02}:
  \bigbreak
  \begin{minipage}{\linewidth}
    \centering
    \multiinclude[format=png, start=0, end=4, graphics={height=5cm}]{FTP/ftp_example/Temp}
  \end{minipage}
\end{frame}
\inccounter

\begin{frame}{Alguns Resultados}
  \begin{thm}[Arkin et al.~\cite{Arkin02}]
    Existe um EPTAS para o FTP com distâncias $L_p$ em qualquer espaço de dimensão fixa $\R^d$.
    \pause
    
    \medskip
    O tempo de execução é $O(n \log n) + 2^{O(d(\overeps)^d\log{(\overeps)})}$.
  \end{thm}
\end{frame}

\begin{frame}{Alguns Resultados}
  \begin{thm}[Abel et al.~\cite{Yu17}]
    O FTP é NP-difícil para distância $L_2$ no plano.
  \end{thm}

  \pause
  \begin{thm}[Demaine e Rudoy~\cite{Erik17}]
    O FTP é NP-difícil para distâncias $L_p$, onde $p>1$, em 3D.
  \end{thm}

  \pause
  \begin{thm}[Pedrosa e Silva~\cite{Lu23}]
    O FTP é fortemente NP-difícil para distância $L_1$ em 3D.
  \end{thm}
\end{frame}
